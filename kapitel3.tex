%
%	Theorieteil
%

\pagebreak
\section{NIS2 Exploration}

\onehalfspacing

\subsection{Mapping between NIS2 and CIS Controls 8.1}

To support NIS2 compliance, CISecurity released in April 2025 a mapping for the CIS Controls v8.1 and the NIS2 Directive.\footnote{\textit{CIS (2025)}: CIS Controls v8.1 Mapping to NIS2 Directive. \cite{cisMapNis2}}

CISecurity uses its in-house methodology to map the individual provisions of a given compliance directive, in this case NIS2, to the individual CIS Controls. The mapping sheet covers all CIS Controls and consists of directly mapped entries and unmapped entries from either direction.

The directive items in the mapping refer directly to the current NIS2 Technical Implementation Guidance and can be easily verified in either direction.\footnote{See \textit{ENISA (2025)}: Technical Implementation Guidance. \cite{nis2}}

\subsection{CIS Benchmarks on AKS}

As a first step, we extract the CIS Controls that are covered in the Benchmark from the most current version of the AKS Benchmark:

\begin{table}[ht]
  \caption{CIS Controls in AKS Benchmark v1.7}
    \begin{tabular}{| l | l |}
    \hline
    ID & Description \\
    \hline\hline
    2.5 & Allowlist Authorized Software \\
    \hline
    3.3 & Configure Data Access Control Lists \\
    \hline
    3.10 & Encrypt Sensitive Data in Transit \\
    \hline
    3.11 & Encrypt Sensitive Data at Rest \\
    \hline
    3.12 & Segment Data Processing and Storage Based on Sensitivity \\
    \hline
    4.1 & Establish and Maintain a Secure Configuration Process \\
    \hline
    4.4 & Implement and Manage a Firewall on Servers \\
    \hline
    4.5 & Implement and Manage a Firewall on End-User Devices \\
    \hline
    4.6 & Securely Manage Enterprise Assets and Software \\
    \hline
    4.7 & Manage Default Accounts on Enterprise Assets and Software \\
    \hline
    5 & Account Management \\
    \hline
    5.4 & Restrict Administrator Privileges to Dedicated Administrator Accounts \\
    \hline
    6 & Access Control Management \\
    \hline
    6.8 & Define and Maintain Role-Based Access Control \\
    \hline
    7.5 & Perform Automated Vulnerability Scans of Internal Enterprise Assets \\
    \hline
    7.6 & Perform Automated Vulnerability Scans of External  Enterprise Assets \\
    \hline
    8.1 & Establish and Maintain an Audit Log Management Process \\
    \hline
    8.2 & Collect Audit Logs \\
    \hline
    8.5 & Collect Detailed Audit Logs \\
    \hline
    13.4 & Perform Traffic Filtering Between Network Segments \\
    \hline
    16.5 & Use Up-to-Date and Trusted Third-Party Software Components \\
    \hline
    \end{tabular}%
  \label{tab:aksBenchmark}%
\end{table}%

The CIS Benchmark tools can automate some of the tests for these CIS Controls and thus can automate regular compliance checks.

\pagebreak

\subsection{CIS AKS Benchmarks Mappings}

From the controls covered by the benchmark, we extract the ones that have direct NIS2 implementation mappings:

\begin{table}[ht]
  \caption{NIS2 Mappings in AKS Benchmark}
    \begin{tabular}{| l | l | l | l | l |}
    \hline
    ID & Type & Item & Description & Directive \\
    \hline\hline
    3.3 & Subset & 11.1 & Access control policy & 11.1.1 \\
    \hline
    3.12 & Subset & 11.4 & Administration systems & 11.4.2 \\
    \hline
    4.1 & Subset & 6.3 & Configuration management & 6.3.1-3 \\
    \hline
    5.4 & Subset & 11.3 & Authentication & 11.3.2 \\
    \hline
    & Subset & 11.6 & Authentication & 11.6.2 \\
    \hline
    6.8 & Superset & 1.2 & Roles, responsibilities and authorities & 1.2.6 \\
    \hline
    & Subset & 11.1 & Access control policy & 11.1.1 \\
    \hline
    & Subset & 11.2 & Access control policy & 11.2.3 \\
    \hline
    & Subset & 11.3 & Access control policy & 11.3.3 \\
    \hline
    7.5, 7.6 & Subset & 6.10 & Vulnerability handling and disclosure & 6.10.2 \\
    \hline
    8.1 & Equivalent & 3.2 & Monitoring and logging & 3.2.1 \\
    \hline
    & Superset & 3.2 & Monitoring and logging & 3.2.2, 3.2.7 \\
    \hline
    8.2 & Subset & 3.2 & Monitoring and logging & 3.2.1 \\
    \hline
    8.5 & Subset & 3.2 & Monitoring and logging & 3.2.3 \\
    \hline
    \end{tabular}%
  \label{tab:aksMapping}%
\end{table}%

\pagebreak

\subsection{NIS2 Article 21}

The NIS2 directive itself is a legal document organized into Chapters and Articles. It has a pan-European scope and targets critical businesses. NIS2 applies to many entities that provide essential services to the European economy and society, such as Operators of Essential Services (OES) and Digital Service Providers (DSPs). While NIS2 primarily targets medium and large enterprises, smaller entities may still be affected, depending on the services they provide.

As we've seen, much of the content concerns reporting requirements and EU-wide cooperation and institutions, which we will not analyze further in this paper.\footnote{See \textit{EU (2022)}: NIS2 Directive. \cite{nis2}}

Article 21, however, defines the required Cybersecurity risk-management measures for critical infrastructure. In paragraph 2, point (g), NIS2 calls for basic cyber hygiene practices and cybersecurity training.

Any entity covered by the directive must, thus, have an IT Security Policy to implement basic cyber hygiene.\footnote{See \textit{NIS 2 Compliant.org (2024)}: List of Policies Required by NIS 2 Directive. \cite{nisPols}} To strengthen their security postures, entities could rely on one of the major frameworks and standards, such as the NIST SP 800 series, ISO/IEC 27001, Mitre Att\&ck, or CIS Controls.\footnote{See \textit{NIS 2 Compliant.org (2024)}: Requirements Checklist for NIS 2. \cite{nisReqs}}

The directive does not mandate whether an entity chooses one of these frameworks or creates its policies. It also does not favor or champion any of the mentioned frameworks.

For this paper, we will focus on CIS Controls and the baselines provided by the CIS Benchmarks to fulfill Article 21's requirements.

\subsection{CIS Control 04}

The CIS Controls in the current version (v8) consist of 18 controls:

\begin{enumerate}
    \item Inventory and Control of Enterprise Assets
    \item Inventory and Control of Software Assets
    \item Data Protection
    \item Secure Configuration of Enterprise Assets and Software
    \item Account Management
    \item Access Control Management
    \item Continuous Vulnerability
    \item Audit Log Management
    \item Email and Web Browser Protections
    \item Malware Defenses
    \item Data
    \item Network Infrastructure
    \item Network Monitoring and Defense
    \item Security Awareness and Skills Training
    \item Service Provider
    \item Application Software Security
    \item Incident Response
    \item Penetration Testing\footnote{See \textit{CIS (2024)}: Critical Security Controls v8. \cite{cisControls}}
\end{enumerate}

As with the NIS2 articles, many of these controls are procedural and cannot be automated. Control 04, however, Secure Configuration of Enterprise Assets and Software, is of particular importance for individual IT systems, such as Kubernetes clusters.

Control 04 consists of several subcontrols:

\begin{enumerate}
    \item Establish and Maintain a Secure Configuration Process
    \item Establish and Maintain a Secure Configuration Process for
Network Infrastructure
    \item Configure Automatic Session Locking on Enterprise Assets
    \item Implement and Manage a Firewall on Servers
    \item Implement and Manage a Firewall on End-User Devices
    \item Securely Manage Enterprise Assets and Software
    \item Manage Default Accounts on Enterprise Assets and Software
    \item Uninstall or Disable Unnecessary Services on Enterprise Assets
and Software
    \item Configure Trusted DNS Servers on Enterprise Assets
    \item Enforce Automatic Device Lockout on Portable End-User Devices
    \item Enforce Remote Wipe Capability on Portable End-User Devices
    \item Separate Enterprise Workspaces on Mobile End-User Devices\footnote{See \textit{CIS (2024)}: Critical Security Controls v8 - Control 04. \cite{cisControls}}
\end{enumerate}

Control 4.1 strongly emphasizes the configuration process, as the default configuration for enterprise software is typically geared toward ease of deployment and use rather than security. This is where the CIS Benchmarks come into play.

\subsection{NIS2 Article 21 and CIS Control 04}

Control 4.1 is the prime candidate to support and evaluate NIS2 Article 21 2(g) as it covers basic cyber hygiene and IT security policy.

The requirements in Article 21 are much broader than those of CIS Control 04, but a secure configuration process is a fundamental element.

Fulfilling CIS Control 04 does not imply compliance with Article 21, but noncompliance with CIS Control 04 does imply noncompliance with Article 21.

The CIS Benchmarks can indicate whether an IT system is potentially NIS2 compliant. Evaluating and running the CIS Benchmarks regularly will support NIS2 compliance checking and defense against cyber attacks.

\subsection{CIS Benchmarks and CIS Control 04}

Maintaining a Secure Configuration Process is important. Leaving default configurations unchecked might result in some of these security risks:

\begin{itemize}
    \item Exposed Services and Ports
    \item Default Accounts and Passwords
    \item Pre-configured DNS Settings
    \item Older or Vulnerable Protocols
    \item Pre-installed Unnecessary Software\footnote{See \textit{CIS (2024)}: Ibid. \cite{cisControls}}
\end{itemize}

To mitigate these, the CIS Benchmarks provide a security baseline for enterprise software, such as Kubernetes, covering all aspects of CIS Control 4.1 and a matching IT Security Policy.\footnote{See \textit{CIS (2024)}: CIS Benchmarks List. \cite{cisBenchmarks}}

There are CIS Benchmarks tailored to the currently available and supported version of Kubernetes and the major hosted versions, such as the Azure Kubernetes Engine.

In the next chapter, we will conduct our experiment and examine the benchmark results from an AKS cluster.
