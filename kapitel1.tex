%
%	Einführung
%

\pagebreak
\section{Introduction}

\onehalfspacing

\subsection{Cyber Security}

Cybersecurity, also called information security or IT security, is the practice of protecting systems, networks, and programs from digital attacks. It aims to reduce the risk of these attacks and prevent the unauthorized exploitation of systems, networks, and technology.

Cybersecurity is an ongoing process because threats and attack vectors are constantly evolving. Organizations and individuals must proactively implement security measures and stay up-to-date on the latest threats.\footnote{See \textit{Gemini (2024)}: What is Cyber Security. \cite{bardCybersec}}

The CIA-Triad defines the key components of Information Security:

\begin{itemize}
    \item Confidentiality
    \item Integrity
    \item Availability\footnote{See \textit{Washington University (2025)}: The CIA Triad. \cite{ciaTriad}} 
\end{itemize}

To address the availability of public infrastructure, our government has introduced the KRITIS designation, expanded the coverage of the BSI Act, and introduced the KRITIS-DachG.\footnote {See \textit{BSI (2024)}: What are Critical Infrastructures. \cite{whatKritis}}

On a European level, NIS2 codifies this.

A lot has changed since the first version of the paper, especially regarding the mappings between CIS Controls and NIS2. The changes made it necessary to revisit the experiment and reevaluate the results.\footnote{See \textit{Frank, C. (2024)}: NIS2 and CIS Controls. \cite{previousPaper}}

\subsection{NIS2}

NIS2, which stands for Network and Information Systems Directive II, is the EU's legislative act to strengthen cybersecurity across the European Union. It's essentially an update to the original NIS Directive implemented in 2016.

It sets stricter requirements for various sectors to improve security for essential entities. These entities include organizations in critical sectors like energy, transport, waste management, healthcare, and digital infrastructure providers.\footnote{See \textit{NIS 2 Compliant.org (2024)}: Comprehensive Guide to the NIS 2 Directive. \cite{nisGuide}}

Compared to the original directive, NIS2 applies to a broader range of businesses and organizations; it recognizes the importance of securing the supply chain and includes measures to address vulnerabilities in third-party vendors and suppliers.

It also emphasizes a risk-based approach to cybersecurity. Organizations must identify and assess their security risks and implement appropriate mitigation measures. Classical tools from Business Continuity Management, such as Risk Impact Analysis and Business Impact Analysis, are now essential parts of the cybersecurity toolkit.\footnote{See \textit{BSI (2025)}: Business Impact Analysis. \cite{bsiBCM}}

NIS2 entered into force on 16 January 2023, and the Member States had 21 months, until 17 October 2024, to transpose its measures into national law.\footnote{See \textit{Negreiro-Achiaga, M. (2023)}: The NIS2 Directive. \cite{nisBrief}}

As of the time of writing, the German government has just presented a proposal to the legislature to codify NIS2 into national law.\footnote{See \textit{BSI (2025)}: NIS-2-Regierungsentwurf. \cite{presseNis2}}

\subsection{CIS Controls and Benchmarks}

CIS Controls, or CIS Critical Security Controls, are a prioritized set of best practices designed to improve an organization's cybersecurity posture. Developed by the \href{https://www.cisecurity.org/}{Center for Internet Security} (CIS), a non-profit organization, these controls are a widely trusted framework for defending against cyberattacks.\footnote{See \textit{CIS (2024)}: Critical Security Controls. \cite{cisControls}}

CIS Benchmarks are configuration guides intended to harden various IT systems against cyberattacks based on the CIS' Critical Security Controls. They provide a set of best practices for securing specific operating systems, applications, and cloud platforms. Many of these Benchmarks have specific versions tied to the underlying software version.\footnote{See \textit{CIS (2024)}: CIS Benchmarks List. \cite{cisBenchmarks}}

Most Kubernetes platform vendors provide automated tools to check against these benchmarks and report and enforce compliance as a first line of defense.

\subsection{Kubernetes}

Kubernetes, or K8s, is an open-source system designed to automate deploying, scaling, and managing applications built using containers. Containers package software in a standardized unit that includes all dependencies the software needs to run, like code, libraries, and settings. This makes them portable and efficient.

Kubernetes helps manage these containers by grouping them logically. This makes it easier to track and manage complex applications with many containers. The original inspiration for Kubernetes came from Google's internal container orchestration system, Borg.\footnote{See \textit{Gemini (2024)}: What is Kubernetes. \cite{bardKubernetes}} 

In 2015, Kubernetes reached the 1.0 milestone, and in 2016, it was donated to the CNCF; the current release of Kubernetes is 1.33, codenamed Octarine. The cutest release was 1.30, codenamed Uwubernetes.

"For the people who built it, for the people who release it, and for the furries who keep all of our clusters online, we present to you Kubernetes v1.30: Uwubernetes, the cutest release to date."\footnote{\textit{Dsouza, A. (2024)}: Kubernetes 1.30. \cite{uwubernetes}}

\begin{figure}[H]
\centering
\caption {Kubernetes 1.30 Release Logo}
\includegraphics[width=0.3\linewidth]{images/k8s-1.30.png}
\label{fig:uwubernetes}
\end{figure}

\subsection{Research Question \& Method}

This paper will examine the new NIS2 security standards, compare them to the current CIS v8 controls, and attempt to map them with a focus on using the CSI Benchmarks. Once we've established a helpful relationship, we will evaluate using CSI benchmarks to check Kubernetes clusters for NIS2 compliance. We will also check whether CIS Benchmarks can provide information about the level of compliance for Kubernetes clusters and offer actionable insights into reaching NIS2 compliance for the relevant controls.

To do this, we will perform an Experiment and check the findings of the Kubernetes CIS Benchmarks on a generic Kubernetes cluster.\footnote{See \textit{Genau, L. (2020)}: Ein Experiment in deiner Abschlussarbeit durchführen. \cite{expScribbr}} For this experiment, we will select Microsoft's Azure Kubernetes Service (AKS) as the target Kubernetes implementation.

The goal of this paper is to determine if CIS Kubernetes Benchmark results can indicate whether a Kubernetes cluster is NIS2-compliant on the platform level.

\subsection{Gender-neutral Pronouns}

Our society is becoming more open, inclusive, and gender-fluid, and now I think it's time to think about using gender-neutral pronouns in scientific texts, too. Two well-known researchers, Abigail C. Saguy and Juliet A. Williams, both from UCLA, propose to use the singular they/them instead: "The universal singular they is inclusive of people who identify as male, female, or nonbinary."\footnote{\textit{Saguy, A. (2020)}: Why We Should All Use They/Them Pronouns. \cite{pronouns}} The aim is to support an inclusive approach in science through gender-neutral language. 

In this paper, I'll attempt to follow this suggestion and invite all my readers to do the same for future articles. Thank you!

If you're not sure about the definitions of gender and sex and how to use them, have a look at the definitions by the American Psychological Association.\footnote{See \textit{APA (2021)}: Definitions Related to Sexual Orientation. \cite{apaDefinitions}}

\subsection{Climate Emergency}

As Professor Rahmstorf puts it: "Without immediate, decisive climate protection measures, my children currently attending high school could already experience a 3-degree warmer Earth. No one can say exactly what this world would look like—it would be too far outside the entire experience of human history. But almost certainly, this earth would be full of horrors for the people who would have to experience it."\footnote{\textit{Rahmstorf, A. (2024)}: Climate and Weather at 3 Degrees More. \cite{3dgreesMore}}
