%
%	Fazit
%

\pagebreak
\section{Summary}

\onehalfspacing

The Center of Internet Security offers benchmarks for testing a Kubernetes cluster against. Aqua Security provides the kube-bench implementation for the CIS Benchmark scans and can support mitigating the findings. Major on-premise Kubernetes platforms, such as SUSE's Rancher, integrate kube-bench and CIS Benchmark scans into their offering.

The CIS Benchmarks cover multiple sections of the CIS Controls v8.1, NIS2 Technical Implementation Guidance, and Article 21 2(g) of the Network and Information Systems Directive II.

From our experiment and analysis, we conclude that the CIS Benchmarks will help enterprises fulfill and monitor the requirements for basic cyber hygiene as outlined in Article 21 2(g). The CIS Benchmarks can help determine whether a Kubernetes cluster is NIS2-compliant on the platform level and are a good complement to ENISA's Technical Implementation Guidance.

Upcoming legislation, such as the KRITIS-DachG ("Gesetz zur Umsetzung der Richtlinie (EU) 2022/2557 und zur Stärkung der Resilienz von Betreibern kritischer Anlagen"), together with the NIS-2-Regierungsentwurf, will further strengthen the security posture of IT systems in Germany and the EU.

Regardless of future developments, governance, risk management, and regulatory compliance will remain critical topics in Enterprise IT, and the CIS Benchmarks are a good tool for monitoring IT systems compliance.

The Terraform plan files for the downstream cluster used in this paper are on my \href{https://github.com/chfrank-cgn/Rancher}{GitHub}.
