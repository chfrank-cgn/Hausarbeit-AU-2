%
%	Fazit
%

\pagebreak
\section{Summary}

\onehalfspacing

The Center of Internet Security offers benchmarks against which to test a Kubernetes cluster. Rancher has integrated CIS scans and can mitigate the findings.

The CIS Benchmarks cover CIS Controls v8 4.1, which corresponds to Article 21 2(g) of the Network and Information Systems Directive II.

As NIS2 matures, we hope to see more formal mappings with CIS Controls v8 and its Implementation Groups to ease practical implementation.

For now, we can conclude that the CIS Benchmarks will help enterprises to fulfill and monitor the requirements for basic cyber hygiene as outlined in Article 21 2(g).

Upcoming legislation, such as the KRITIS-DachG ("Gesetz zur Umsetzung der Richtlinie (EU) 2022/2557 und zur Stärkung der Resilienz von Betreibern kritischer Anlagen"), will further strengthen the security posture of IT systems in the EU.

Regardless of future developments, governance, risk management, and regulatory compliance will remain critical topics in Enterprise IT, and the CIS Benchmarks are a good tool for monitoring IT systems compliance.

The Terraform plan files for the downstream clusters used in this paper are on my \href{https://github.com/chfrank-cgn/Rancher}{GitHub}.

Happy Ranching!
